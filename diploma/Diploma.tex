% Nejprve uvedeme tridu dokumentu s volbami
\documentclass[bc,male,java,dept460]{diploma}						% jednostranny dokument
%\documentclass[bc,male,java,dept460,twoside]{diploma}		% oboustranny dokument
\usepackage[utf8]{inputenc}
\usepackage[czech]{babel}

\newcommand{\iic}{I\textsuperscript{2}C }
\newcommand{\iis}{I\textsuperscript{2}S }


% Zadame pozadovane vstupy pro generovani titulnich stran.
\ThesisAuthor{Bc. Pavel Kovář}

% U bakalarske praxe neni nutne nazev zadavat
\ThesisTitle{Modul USB FM rádia}

% U bakalarske praxe neni nutne anglicky nazev zadavat
\EnglishThesisTitle{USB FM Radio Modul}

\SubmissionDate{1. dubna 2016}

\PrintPublicationAgreement{true}

\Thanks{Rád bych na tomto místě poděkoval všem, kteří mi s prací pomohli, protože bez nich by tato práce nevznikla. (Uprav)}

\CzechAbstract{Tato práce popisuje návrh USB FM přijímače se dvěma tunery. Jeden tuner slouží pro přehrávání zvuku a druhý pro vyhledávání dalších stanic. přijímač je v systému reprezentován jako USB zvuková karta.\\ 
Příjem je realizován dvojicí integrovaných obvodů Si4735-DU. Tyto jsou přes \iis a \iic  spojeny s MCU PIC32MX250F128B, který přes USB zajišťuje komunikaci s počítačem. V rámci firmware MCU je, po neúspěchu s Microchip harmony frameworkem, napsán vlastní USB stack.\\
Knihovna je napsána v jazyku C s využitím knihovny libusb. Poskytuje funkce pro tři úrovně přístupu k tunerům.\\
Demonstrační aplikace je ve formě grafického uživatelského rozhraní, napsaná  v C++ s využitím QT frameworku.\\
Vše je funkční pod OS Linux i Windows.}

\CzechKeywords{FM rádio, USB, RDS, QT, libusb, PIC}

\EnglishAbstract{This work describes design of USB FM radio receiver with two tuners. One tuner is for radio playback, second one seeks new stations. In computer, device acts as sound card.\\
Receiving is done by couple of Si4735-DU integrated circuits, which are connected to MCU via \iic and \iis. MCU forwards data over USB to computer and back. Use of Microchip harmony framework was not successful so in firmware is USB stack written from scratch.\\
Library is written in C with use of libusb library. There are three levels of functions to access tuners.\\
Demo application has graphical user interface and is written in C++ in QT framework.\\
All works under Linux and Windows.}

\EnglishKeywords{FM radio receiver, USB, RDS, QT, libusb, PIC}

% Pridame pouzivane zkratky (pokud nejake pouzivame).
\AddAcronym{MCU}{Microcontroller unit}
\AddAcronym{\iic}{Inter-Integrated Circuit}
\AddAcronym{\iis}{Integrated Interchip Sound}
\AddAcronym{USB}{Universal Serial Bus}
\AddAcronym{PCM}{Pulse-code modulation}
\AddAcronym{RDS}{Radio Data System}


% Zadame cestu a jmeno souboru ci nekolika souboru s digitalizovanou podobou zadani prace
% Pri sazbe se pak hledaji soubory Figures/Zadani1.jpg, Figures/Zadani2.jpg atd.
% Do diplomove prace se postupne vlozi vsechny existujici soubory Figures/ZadaniXXX.jpg
% Pokud toto makro zapoznamkujeme sazi se stranka s upozornenim
%\ThesisAssignmentImagePath{Figures/Zadani}

% Zadame soubor s digitalizovanou podobou prohlaseni
% Pokud toto makro zapoznamkujeme sazi se cisty text prohlaseni
%\DeclarationImageFile{figures/Prohlaseni.jpg}



% Zacatek dokumentu
\begin{document}

% Nechame vysazet titulni strany.
\MakeTitlePages

% Asi urcite budeme potrebovat obsah prace.
\tableofcontents
\cleardoublepage	% odstrankujeme, u jednostranneho dokumentu o jednu stranku, u oboustrenneho o dve

% Jsou v praci tabulky? Pokud ano vysazime jejich seznam.
% Pokud ne smazeme nasledujici makro.
\listoftables
\cleardoublepage	% odstrankujeme, u jednostranneho dokumentu o jednu stranku, u oboustrenneho o dve

% Jsou v praci obrazky? Pokud ano vysazime jejich seznam.
\listoffigures
\cleardoublepage	% odstrankujeme, u jednostranneho dokumentu o jednu stranku, u oboustrenneho o dve


% Jsou v praci vypisy programu? Pokud ano vysazime jejich seznam.
\lstlistoflistings
\cleardoublepage	% odstrankujeme, u jednostranneho dokumentu o jednu stranku, u oboustrenneho o dve



% Zacneme uvodem
\section{Úvod}
\label{sec:Uvod}
Tento text je ukázkou sazby diplomové práce v \LaTeX{}u pomocí třídy dokumentů \verb|diploma|.
Pochopitelně text není skutečnou diplomovou prací, ale jen ukázkou použití
implementovaných maker v praxi. V kapitole \ref{sec:Typo} jsou ukázky použití
různých maker a prostředí. V kapitole \ref{sec:Conclusion} bude \uv{jako závěr}. Zároveň
tato kapitola slouží jako ukázka generování křížových odkazů v \LaTeX{}u.

\section{Závěr}
\label{sec:Conclusion}

\InsertFigure{figures/breadborad-landscape.jpg}{123mm}{Modul v kontaktním poli.}{fig:breadborad}

Cílem této práce byl návrh modulu USB FM přijímače, který bude v operačním systému reprezentován jako zvuková karta. Přijímač má obsahovat dva tunery, kdy jeden bude používán pro~přehrávání a druhý pro vyhledávání dalších stanic. 

Prvními dvěma úkoly byl výběr vhodných součástek a ověření funkce na kontaktním poli. Toto je rozebráno v kapitole \ref{sec:Vyber}. Byly vybrány tunery ve formě integrovaných obvodů SI4735-D60, které budou na sběrnici USB připojeny prostřednictvím mikrokontroléru PIC32MX250F128B. Výslednou konstrukci nejlépe ilustruje obrázek \ref{fig:tuner-block} na straně \pageref{fig:tuner-block}. Do mikrokontroléru byl napsán firmware s vlastním zjednodušeným ovladačem USB. Nad tímto ovladačem běží USB - \iic tunel, pomocí kterého jsou ovládány tunery a rozhraní, které přeposílá digitální zvuk z tunerů do USB. Funkčnost vzniklého zařízení jsem úspěšně ověřil na kontaktním poli (viz. obrázek \ref{fig:breadborad}). Nad rámec zadání jsme navíc navrhl  desku plošných spojů na které jsem zařízení úspěšně oživil (fotografie v příloze~\ref{sec:ap-pcb}). Zařízení má v této podobně mnohem kvalitnější příjem. Schéma zapojení je v příloze \ref{sec:ap-schema}. Zároveň je v elektronické verzi spolu s podklady pro výrobu DPS umístěn na~přiloženém CD.

Třetím úkolem bylo napsat knihovnu pro práci s tunerem pod OS Linux a Windows. Knihovna byla napsána v jazyce C s využitím multiplatformní knihovny libusb. Knihovna je rozdělena do třech vrstev a umožňuje tak snadné rozšíření funkcionality. Knihovna v současné podobně umožňuje přeladění vybraného tuneru na danou frekvenci, vyhledávání stanic a vyčtení parametrů naladěné stanice. Knihovna také byla doplněna o možnost příjmu RDS dat. Způsob jakým knihovna ovládání tunerů knihovnou je popsáno v kapitole \ref{sec:tuner}. Samotná knihovna je popsána v kapitole \ref{sec:knihovna}, samotné rozhraní knihovny je podrobně popsáno v dokumentaci vygenerované nástrojem Doxygen, která je umístěna na přiloženém CD.

Čtvrtým a poslední úkolem bylo napsat uživatelský program, který bude demonstrovat využití knihovny. Napsal jsem aplikaci s grafickým uživatelským rozhraním s využitím multiplatformního frameworku QT5. Tato aplikace umožňuje ladění frekvencí, vyhledávání stanic, zobrazuje parametry signálu naladěné frekvence a zobrazuje její RDS radio text. Zároveň jak~je~požadováno v úvodu zadání průběžně druhým tunerem vyhledává dostupně stanice a zobrazuje je~v~tabulce. Rozhraní nejlépe ilustruje obrázek \ref{fig:gui} na straně \pageref{fig:gui}.

%\subsection{Omezení}

%TODO:
%\begin{itemize}

%\item Nemožnost dostat se v suspendu do 0,5 mA
%\item Test mód neimplementován. Nutný pouze pro získání USB loga.

%suspend a test mod
%\end{itemize}
\FloatBarrier
\subsection{Možnosti dalšího rozšíření}
Hardware a částečně i firmware MCU je připraven pro přenos zvuku i z druhého tuneru. Vhodným doplnění deskriptoru USB konfigurace by bylo možno zařízení přepínat do režimu dvou zvukových karet. Poté by bylo možné nahrávat jinou stanici při současném přehrávání jiné a~nebo využít zařízení pro~streaming několika rozhlasových stanic.  

Na desce plošných spojů je připraveny pozice pro dvě tlačítka, která zatím nejsou osazena. Drobnou úpravou firmware by bylo možné využít je k vyhledávání stanic bez potřeby jakéhokoliv dalšího software, popřípadě stisky tlačítek posílat přes USB do aplikace a tam je dále vyhodnocovat.

Tunery umožňují příjem rozhlasu i na středních a dlouhých vlnách. Toto by vyžadovalo jednak úpravu knihovny, která tunery přepne do režimu pro příjem těchto pásem a také bude třeba hardware doplnit o vhodnou anténu spolu s transformátorem viz. katalogový list \cite{tuner-datasheet}.

Uživatelská aplikaci by bylo vhodné doplnit minimálně o možnost ovládat přehrávání zvuku a popřípadě jeho záznam.





\bigskip
\begin{flushright}
Bc. Pavel Kovář
\end{flushright}

\begin{thebibliography}{99}

\bibitem{goossens94} Goossens, Michel,
\textit{The \LaTeX\ companion,} New York: Addison, 1994.

\bibitem{lamport94} Lamport, Leslie,
\textit{\LaTeX: a document preparation system: user's guide and reference manual},
New York: Addison-Wesley Pub. Co., 1994.

\end{thebibliography}


\appendix
\section{Grafy a měření}
Tohle je příloha k práci. Většinou se sem dávají grafy, tabulky, které by vzhledem
ke svému počtu překážely v textu diplomky. (Upravit)
\clearpage

%\InsertFigure{figures/Graf}{0.7\textwidth}{Nějaký graf}{fig:SampleGraph}


\end{document}
