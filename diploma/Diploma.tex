% Nejprve uvedeme tridu dokumentu s volbami
\documentclass[bc,male,java,dept460]{diploma}						% jednostranny dokument
%\documentclass[bc,male,java,dept460,twoside]{diploma}		% oboustranny dokument
\usepackage[utf8]{inputenc}
\usepackage[czech]{babel}

\newcommand{\iic}{I\textsuperscript{2}C }
\newcommand{\iis}{I\textsuperscript{2}S }


% Zadame pozadovane vstupy pro generovani titulnich stran.
\ThesisAuthor{Bc. Pavel Kovář}

% U bakalarske praxe neni nutne nazev zadavat
\ThesisTitle{Modul USB FM rádia}

% U bakalarske praxe neni nutne anglicky nazev zadavat
\EnglishThesisTitle{USB FM Radio Modul}

\SubmissionDate{1. dubna 2016}

\PrintPublicationAgreement{true}

\Thanks{Rád bych na tomto místě poděkoval všem, kteří mi s prací pomohli, protože bez nich by tato práce nevznikla. (Uprav)}

\CzechAbstract{Tato práce popisuje návrh USB FM přijímače se dvěma tunery. Jeden tuner slouží pro přehrávání zvuku a druhý pro vyhledávání dalších stanic. přijímač je v systému reprezentován jako USB zvuková karta.\\ 
Příjem je realizován dvojicí integrovaných obvodů Si4735-DU. Tyto jsou přes \iis a \iic  spojeny s MCU PIC32MX250F128B, který přes USB zajišťuje komunikaci s počítačem. V rámci firmware MCU je, po neúspěchu s Microchip harmony frameworkem, napsán vlastní USB stack.\\
Knihovna je napsána v jazyku C s využitím knihovny libusb. Poskytuje funkce pro tři úrovně přístupu k tunerům.\\
Demonstrační aplikace je ve formě grafického uživatelského rozhraní, napsaná  v C++ s využitím QT frameworku.\\
Vše je funkční pod OS Linux i Windows.}

\CzechKeywords{FM rádio, USB, RDS, QT, libusb, PIC}

\EnglishAbstract{This work describes design of USB FM radio receiver with two tuners. One tuner is for radio playback, second one seeks new stations. In computer, device acts as sound card.\\
Receiving is done by couple of Si4735-DU integrated circuits, which are connected to MCU via \iic and \iis. MCU forwards data over USB to computer and back. Use of Microchip harmony framework was not successful so in firmware is USB stack written from scratch.\\
Library is written in C with use of libusb library. There are three levels of functions to access tuners.\\
Demo application has graphical user interface and is written in C++ in QT framework.\\
All works under Linux and Windows.}

\EnglishKeywords{FM radio receiver, USB, RDS, QT, libusb, PIC}

% Pridame pouzivane zkratky (pokud nejake pouzivame).
\AddAcronym{MCU}{Microcontroller unit}
\AddAcronym{\iic}{Inter-Integrated Circuit}
\AddAcronym{\iis}{Integrated Interchip Sound}
\AddAcronym{USB}{Universal Serial Bus}
\AddAcronym{PCM}{Pulse-code modulation}
\AddAcronym{RDS}{Radio Data System}


% Zadame cestu a jmeno souboru ci nekolika souboru s digitalizovanou podobou zadani prace
% Pri sazbe se pak hledaji soubory Figures/Zadani1.jpg, Figures/Zadani2.jpg atd.
% Do diplomove prace se postupne vlozi vsechny existujici soubory Figures/ZadaniXXX.jpg
% Pokud toto makro zapoznamkujeme sazi se stranka s upozornenim
%\ThesisAssignmentImagePath{Figures/Zadani}

% Zadame soubor s digitalizovanou podobou prohlaseni
% Pokud toto makro zapoznamkujeme sazi se cisty text prohlaseni
%\DeclarationImageFile{figures/Prohlaseni.jpg}



% Zacatek dokumentu
\begin{document}

% Nechame vysazet titulni strany.
\MakeTitlePages

% Asi urcite budeme potrebovat obsah prace.
\tableofcontents
\cleardoublepage	% odstrankujeme, u jednostranneho dokumentu o jednu stranku, u oboustrenneho o dve

% Jsou v praci tabulky? Pokud ano vysazime jejich seznam.
% Pokud ne smazeme nasledujici makro.
\listoftables
\cleardoublepage	% odstrankujeme, u jednostranneho dokumentu o jednu stranku, u oboustrenneho o dve

% Jsou v praci obrazky? Pokud ano vysazime jejich seznam.
\listoffigures
\cleardoublepage	% odstrankujeme, u jednostranneho dokumentu o jednu stranku, u oboustrenneho o dve


% Jsou v praci vypisy programu? Pokud ano vysazime jejich seznam.
\lstlistoflistings
\cleardoublepage	% odstrankujeme, u jednostranneho dokumentu o jednu stranku, u oboustrenneho o dve



\section{Úvod}
\label{sec:Uvod}
%Tento text je ukázkou sazby diplomové práce v \LaTeX{}u pomocí třídy dokumentů \verb|diploma|.

Cílem této diplomové práce je navrhnout USB FM přijímač rádiového vysílání. Přijímač bude v systému reprezentován jako zvuková karta a bude obsahovat dva tunery. Jeden bude sloužit k samotnému příjmu vysílání a druhý k vyhledávání dalších stanic. Dále napsat knihovnu, umožňující ovládání tuneru pod Operačními systémy Linux a Windows. Text této práce je členěn do třech kapitol.

První kapitola se zabývá konstrukcí modulu. Je zde shrnuto a~zhodnoceno několik možností jak realizovat jednak samotný příjem rozhlasového vysílání a~také několik možností řešení napojení tunerů na USB sběrnici.

Druhá kapitola je věnována popisu vybraného tuneru. Je zde uveden zvolený způsob získání zvukových dat, filozofie ovládání tuneru a popis všech použitých příkazů pro tuneru.

Ve třetí kapitole je rozebráno napojení tunerů na USB sběrnici. Je zde pois USB popis porblémů s harmony , chyby v křemíku použitého MCU, porblémy nesynchronizovaných hodin udesílání zvuku. Popsáno vzniklé USB rozhraní (TODO)

Čtvrtá kapitola se věnuje vzniklé knihovně a demonstračnímu programu. V první části kapitoly je popis filozofie a rozvrstvení knihovny včetně popisu funkcí. Doplněno ukázkovým kódem práce. Závěru kapitoly je popsána demonstrační aplikace. (TODO)

%V závěru jsou shrnuty dosažené cíle, vyzdvihnut vlastní přínos

\section{Závěr}
\label{sec:Conclusion}
%Tak doufám, že Vám tato ukázka k něčemu byla. Další informace najdete v~publikacích
%\cite{goossens94,lamport94}.

TODO:
\begin{itemize}
\item Zmínit že příjem na PCB je mnohem kvalitnější než v nepájivém poli a změřit to pomocí SNR
\end{itemize}

\bigskip
\begin{flushright}
Bc. Pavel Kovář
\end{flushright}

\begin{thebibliography}{99}

\bibitem{goossens94} Goossens, Michel,
\textit{The \LaTeX\ companion,} New York: Addison, 1994.

\bibitem{lamport94} Lamport, Leslie,
\textit{\LaTeX: a document preparation system: user's guide and reference manual},
New York: Addison-Wesley Pub. Co., 1994.

\end{thebibliography}


\appendix
\section{Grafy a měření}
Tohle je příloha k práci. Většinou se sem dávají grafy, tabulky, které by vzhledem
ke svému počtu překážely v textu diplomky. (Upravit)
\clearpage

%\InsertFigure{figures/Graf}{0.7\textwidth}{Nějaký graf}{fig:SampleGraph}


\end{document}
