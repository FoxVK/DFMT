\section{Tuner}
\label{sec:tuner}
Jak jsem zmínil v kapitole \ref{subsec:konstrukce}
\begin{itemize}
\item nepoužití přerušení a důsledky - nutnost polingu
\item nějak vkusně s ohledem na předchozí kapitolu vyzdvnhnout použítí \iic - jendou větou na začátek
\end{itemize}


\subsection{Zvukové rozhraní \iis}

\begin{table}[ht!]
\begin{center}
\begin{tabular}{|l|c|l|l|}
\hline 
MCU & Směr & Tuner & Význam \\ 
\hline 
SDI & $\Rightarrow$ & DOUT & Datový signál \\ 
\hline 
SCK & $\Rightarrow$ & DCLK & Hodiny \\ 
\hline
SS & $\Leftarrow$ & DFS & Signál určení kanálu \\ 
\hline 
\end{tabular}
\end{center}
\caption{Popis \iis signálů.}
\label{tab:iis_signals} 
\end{table}


Rozhraní \iis je v případě jednosměrného přenosu v podstatě rozhraní SPI doplněné o další datový signál. Význam signálů včetně odpovídajících názvů pinů na MCU a tuneru shrnuje tabulka \ref{tab:iis_signals}. V případě \iis se nejedná o sběrnici. Vždy komunikují právě dvě zařízení kde jedno je master a druhé slave. Master vždy vysílá datový signál a signál přepínání kanálů. Datový signál vysílá samozřejmě zřízení, které je zdrojem zvuku. Je možný i obousměrný přenos, potom je nutné použití dvou datových linek.

Tuner umí pracovat pouze v režimu slave, takže hodinový signál a signál přepínání kanálů je generován z MCU. Vzorkování zvuku v tuneru je řízeno jeho vlastním interním oscilátorem. Díky tomuto pravděpodobně občas dojde k zahození některého vzorku pokud MCU čte pomaleji než tuner provádí vzorkování, a nebo dojde k přečtení náhodných dat v opačném případě. Nicméně v praxi jsem s tímto nezaznamenal žádný problém.  

\InsertFigure{figures/i2s_format.png}{120mm}{Časový diagram \iis přenosu. (Převzato z \cite{iis})}{fig:iis-diagram}

Vyjma standardního formátu dat popsaného v \iis specifikace \cite{iis} vzniklo ještě několik dalších formátů. Použil jsem standardní formát, tak jak je zobrazen na obrázku \ref{fig:iis-diagram}.

TODO:
\begin{itemize}
\item Umravnit pozicování obrázku
\item Problém synchronizace hodin - presunout do USB jako samostatnou podkapitolu a nebo spáchat samostatnou kapitolu přeposílání zvuku
\end{itemize}


\subsection{Ovládací rozhraní \iic}
Ovládání tuneru se provádí přes rozhraní \iic. Toto rozhraní je součásti drtivé většiny mikrokontrolérů, není ho třeba blíže popisovat. Tuner se do režimu \iic přepne připojením pinů GPO1 na 1 GPO2 na 0.

\begin{table}[ht!]
\begin{center}
\begin{tabular}{|c|c|c|}
\hline 
SEN & Adresa čtení & Adresa zápisu  \\ 
\hline 
0 & 0x23 & 0x22 \\ 
\hline 
1 & 0xC7 & 0xC6 \\ 
\hline 
\end{tabular}
\end{center}
\caption{\iic adresy.}
\label{tab:iic_addresses} 
\end{table}

Základní informací nezbytnou pro komunikace s tunerem je \iic adresa. Tuner umožňuje změnu adresy pomocí změny úrovně na pinu SEN. Adresy jsou v tabulce \ref{tab:iic_addresses}.

\subsection{Ovládání tuneru}
\label{subsec:tuner-control}
Tuner se ovládá zapisováním příkazů a případným čtením odpovědí. Dále je zde dvojice příkazů pro čtení a zápis proměnných. (properties) tuneru. Kompletní popis všech příkazů a proměnných je v programovací příručce k tuneru \cite{tuner-programing}.

\begin{table}[ht!]
\begin{center}
\begin{tabular}{|l|c|l|l|}
\hline 
Název pole & Délka &  \\ 
\hline
CMD & 1 B & Příkaz.\\
\hline
ARG1 - ARG7 & 0 - 7 B & Případný argument příkazu.\\
\hline
\end{tabular} 
\end{center}
\caption{Formát příkazu pro tuner.}
\label{tab:tuner-cmd} 
\end{table}

\begin{table}[ht!]
\begin{center}
\begin{tabular}{|l|c|l|l|}
\hline 
Název pole & Délka &  \\ 
\hline
STATUS & 1 B & Status tuneru.\\
\hline
RESP1 - RESP15 & 0 - 15 B & Případná další data odpovědi.\\
\hline
\end{tabular} 
\end{center}
\caption{Formát odpovědi tuneru.}
\label{tab:tuner-rpl} 
\end{table}

Všechny příkazy mají jednotný formát vyobrazený v tabulce \ref{fig:tuner-cmd}. Každý příkaz je jednou transakcí zápisu na \iic sběrnici. Případnou odpověď na příkaz je možné získat následnou transakcí čtení z \iic sběrnice. Formát odpovědi je v tabulce \ref{tab:tuner-rpl}. Jak je patrné v odpověď na každý příkaz obsahuje vždy minimálně jeden byte se stavem zařízení (STATUS). Význam jednotlivých je následující:

\begin{itemize}
\item Bit 7 \textbf{CTS} - Pokud je v 1 je možné do tuneru odeslat další příkaz.
\item Bit 6 \textbf{ERR} - Pokud je v 1 signalizuje chybu provádění příkazu.
\item Bity 5 - 3 \textbf{Rezervováno} - Hodnoty těchto bitů se může měnit.
\item Bit 2 \textbf{RDSINT} - Pokud je v 1 signalizuje vznik přerušení od přijmu RDS.
\item Bit 1 \textbf{ASQINT} - Pokud je v 1 signalizuje dokončení měření parametrů přijmu.
\item Bit 0 \textbf{STCINT} - Pokud je v 1 signalizuje dokončení ladění nebo vyhledávání stanice.
\end{itemize}  

Bity CTS a ERR se aktualizují vždy. Pro čtení nejnižších třech bitů je potřeba aktualizovat jejich hodnoty příkazem GET\_INT\_STATUS. Tento příkaz má číslo 0x14 a nemá žádné argumenty.

Jak jsem zmínil výše, kromě příkazů se parametry tuneru nastavují a nebo čtou pomocí properties. jedná se vždy o 16-ti bitové hodnoty, které se identifikují taktéž 16-ti bitovým číslem. 

Zápis hodnoty se provádí příkazem SET\_PROPERTY (0x12), za kterým následuje jeden nulový byte, za ním horní a poté dolní byte čísla property. Dále se vyšle horní byte a poté dolní byte vlastní hodnoty. 

Čtení se provádí obdobně, příkazem GET\_PROPERTY (0x14) za kterým se vyšle nulový byte následovaný horním a dolním byte čísla zapisované property. Hodnota si získá následným čtením z \iic sběrnice. Horní byte hodnoty je až ve druhém bytu argumentu ARG-2 (viz. tab. \ref{tab:tuner-rpl}), dolní byte v je v ARG-3.

%TODO přidat tabulky pro property ?

\subsubsection{Inicializace tuneru}
\InsertFigure{figures/tuner-init.eps}{111mm}{Diagram inicializace tuneru.}{fig:tuner-init}

Po spuštění MCU vždy provádí inicializaci obou tunerů. Průběh inicializace shrnuje diagram \ref{fig:tuner-init}. Z důvodu zkrácení doby spuštění modulu probíhá inicializace formou jednoduchého "multitaskingu". Pokud je jeden tuner zaneprázdněn testuje se stav druhého tuneru a v případě, že není zaneprázdněn dojde k vyslání příkazu, pokud je zaneprázdněn testuje se znova první  tuner a tak dále. Ze stejného důvodu je v příkazu FM\_TUNE\_STATUS nastaven příznak zrychleného nepřesného ladění.


\begin{table}[ht!]
\begin{center}
\begin{tabular}{|c|c|c|c|c|c|c|c|c|}
\hline 
Bit & 7. & 6. & 5. & 4. & 3. & 2. & 1. & 0. \\ 
\hline 
CMD & \multicolumn{8}{c|}{0x01} \\ 
\hline 
ARG1 & CTSIEN = 0 & GPO2OEN = 0 & PATCH = 0 & XOSCEN = 0 & \multicolumn{4}{c|}{FUNC = 0} \\ 
\hline 
ARG2 & \multicolumn{8}{c|}{OPMODE = 0b10110000} \\ 
\hline 
\end{tabular} 
\end{center}
\caption{Příkaz spuštění tuneru POWER\_UP.}
\label{tab:tuner-power-up} 
\end{table}

Inicializace začíná spuštěním tuneru příkazem POWER\_UP. Formát příkazu je v tabulce \ref{tab:tuner-power-up}. Součástí příkazu jsou tyto parametry:

\begin{itemize}
\item \textbf{CTSIEN} - Možnost hardwarové signalizace přerušení v okamžiku kdy je možné odeslat další příkaz. Hardwarové přerušení nepoužívám.
\item \textbf{GPO2OEN} - Nastavení pinu GPO2 jako výstupu.
\item \textbf{PATCH} - Umožňuje načtení novějšího firmware do integrovaného obvodu tuneru.
\item \textbf{XOSCEN} - Použití krystalu jako zdroje hodinové frekvence. Místo krystalu používám přímo hodinový kmitočet rozhraní \iis.
\item \textbf{FUNC} - Použitá hodnota nastaví režim přijmu FM rádia.
\item \textbf{OPMODE} - Nastavení výstupu zvuku. Použitá hodnota znamená "pouze digitální výstup \iis".
\end{itemize}

Odpovědí na tento příkaz je pouze jeden byte se statusem tuneru.

Dále je potřeba nastavit dvě property. REFCLK\_PRESCALE hodnotu děličky hodinového kmitočtu a do REFCLK\_FREQ výslednou frekvenci, která se musí pohybovat v rozsahu $ 31,130 - 34,406 Khz $. Jak už jsem zmínil jako zdroj hodinové frekvence je použita hodinová frekvence \iis rozhraní. Tato frekvence vyplývá ze součinu počtu kanálů, jejich rozlišení a vzorkovací frekvence. $ 2 \cdot 16 \cdot 48 = 1536 kHz $. Dělící poměr tedy vychází $ 1536 / 32 = 48 $ při frekvenci $ 32000 kHz $.

Zatím co REFCLK\_FREQ obsahuje pouze 16-ti bitovou hodnotu frekvence v Hz v REFCLK\_PRESCALE jsou pro hodnotu dělícího poměru vyhrazeny bity 0-11. Bity 13-15 musí být nulové, bit 12 je nastaven na hodnotu 1, což znamená, že se jako zdroj hodinové frekvence použije \iis.

Poslední property, kterou je potřeba nastavit, je DIGITAL\_OUTPUT\_SAMPLERATE. Obsahuje hodnotu vzorkovací frekvence. V tomto případě je to hodnota 48000.

\begin{table}[ht!]
\begin{center}
\begin{tabular}{|l|c|c|}
\hline 
Název & Číslo & Hodnota \\ 
\hline 
REFCLK\_FREQ & 0x0201 & 0x7D00 \\ 
\hline 
REFCLK\_PRESCALE & 0x0202 & 0x102A \\ 
\hline 
DIGITAL\_OUTPUT\_SAMPLE\_RATE & 0x0104 & 0xBB80 \\ 
\hline 
\end{tabular} 
\end{center}
\caption{Properties použité pro inicializaci tuneru.}
\label{tab:tuner-init-prop} 
\end{table}

Properties se nastavují pomocí příkazu SET\_PROPERTY popsaného v kapitole \ref{subsec:tuner-control}. Výsledné hodnoty a čísla properties jsou shrnuty v tabulce \ref{tab:tuner-init-prop}.

\begin{table}[ht!]
\begin{center}
\begin{tabular}{|c|c|c|c|c|c|c|c|c|}
\hline 
Bit & 7. & 6. & 5. & 4. & 3. & 2. & 1. & 0. \\ 
\hline 
CMD & \multicolumn{8}{c|}{0x20} \\ 
\hline 
ARG1 & \multicolumn{6}{c|}{0x00} & FREEZE = 0 & FAST = 1 \\ 
\hline 
ARG2 & \multicolumn{8}{c|}{FREQ$_{{H}}$ = 0x24} \\ 
\hline 
ARG3 & \multicolumn{8}{c|}{FREQ$_{{L}}$ = 0x94} \\ 
\hline 
ARG4 & \multicolumn{8}{c|}{ANTCAP = 0x00} \\ 
\hline 
\end{tabular} 
\end{center}
\caption{Příkaz spuštění tuneru FM\_TUNE\_FREQ.}
\label{tab:tuner-tune-freq} 
\end{table}

Poslední částí inicializace tuneru je naladění frekvence 93,7 MHz příkazem FM\_TUNE\_FREQ. Formát příkazu včetně použitých hodnot je v tabulce \ref{tab:tuner-tune-freq}. Význam parametrů je následující:

\begin{itemize}
\item \textbf{FREEZE} - Nastavení způsobí pozvolný přechod zvuku po přeladění.
\item \textbf{FAST} - Nastavení způsobí rychlé ale nepřesné přeladění.
\item \textbf{FREQ$_{{H}}$} - Horní byte frekvence v desetinách MHz.
\item \textbf{FREQ$_{{L}}$} - Dolní byte frekvence v desetinách MHz.
\item \textbf{ANTCAP} - Nastavení kapacity vstupního kondenzátoru antény. Hodnota 1-191 pF. Hodnota 0 znamená automatické nastavení.
\end{itemize}

Odpovědí na tento příkaz je pouze jeden byte obsahující status tuneru. Dokončení ladění je signalizováno nastavením bitu STCINT ve statusu. K aktualizaci hodnoty tohoto bytu je nutné vždy vyslat příkaz GET\_INT\_STATUS. Tento příkaz se skládá z jediného byte s číslem příkazu 0x14. Odpovědí na tento příkaz je rovněž jediný byte se statusem tuneru kde jsou aktualizovány hodnoty bitů signalizujících přerušení RDSINT, ASQINT a STCINT.


\begin{table}[ht!]
\begin{center}
\begin{tabular}{|c|c|c|c|c|c|c|c|c|}
\hline 
Bit & 7. & 6. & 5. & 4. & 3. & 2. & 1. & 0. \\ 
\hline 
CMD & \multicolumn{8}{c|}{0x22} \\ 
\hline 
ARG1 & \multicolumn{6}{c|}{0x00} & CANCEL = 0 & INTACK = 1 \\ 
\hline 
\end{tabular} 
\end{center}
\caption{Příkaz zjištění stavu ladění FM\_TUNE\_STATUS.}
\label{tab:tuner-tune-status} 
\end{table}

\begin{table}[ht!]
\begin{center}
\begin{tabular}{|c|c|c|c|c|c|c|c|c|}
\hline 
Bit & 7. & 6. & 5. & 4. & 3. & 2. & 1. & 0. \\ 
\hline 
SATUS & CTS & ERR & X & X & RSQINT & RDSINT & X & STCINT \\ 
\hline 
RESP1 & BLTF & X & X & X & X & X & AFCRL & VALID \\ 
\hline 
RESP2 & \multicolumn{8}{c|}{READFREQ$_{{H}}$} \\ 
\hline 
RESP3 & \multicolumn{8}{c|}{READFREQ$_{{L}}$} \\ 
\hline 
RESP4 & \multicolumn{8}{c|}{RSSI} \\ 
\hline 
RESP5 & \multicolumn{8}{c|}{SNR} \\ 
\hline 
RESP6 & \multicolumn{8}{c|}{MULT} \\ 
\hline 
\end{tabular} 
\end{center}
\caption{Odpověď na příkaz zjištění stavu ladění FM\_TUNE\_STATUS.}
\label{tab:tuner-tune-status-resp} 
\end{table}

Ke smazání bitu STCINT je použit příkaz FM\_TUNE\_STATUS viz. tabulka \ref{tab:tuner-tune-status}. Kromě smazání tohoto bitu nastavením parametru INTACK je možné nastavením parametru INTACK zrušit probíhající ladění nebo vyhledávání stanice. Jak je vidět z tabulky \ref{tab:tuner-tune-status-resp}, odpověď na příkaz obsahuje kromě statusu tuneru následující informace:

\begin{itemize}
\item \textbf{BLTF} - Je nastaven pokud vyhledávání stanice přeteklo přes maximální nebo minimální frekvenci.
\item \textbf{AFCRL} - Je nastaven pokud je automatické dolaďování aktivní.
\item \textbf{VALID} - Naladěná frekvence byla vyhodnocena jako validní. 
\item \textbf{READFREQ$_{{H}}$} - Horní byte naladěné frekvence v desetinách MHz.
\item \textbf{READFREQ$_{{L}}$} - Dolní byte naladěné frekvence v desetinách MHz.
\item \textbf{RSSI} - Indikátor síly přijímaného signálu v dBuV. %TODO přidej mikro
\item \textbf{SNR} - Odstup signálu od šumu v dB.
\item \textbf{MULT} - Indikátor míry odrazů v signálu.
\end{itemize}


\subsubsection{Čtení RDS z tuneru}
TODO:
\begin{itemize}
\item vyčtení z tuneru
\item dekódování základních informací- třeba radio text
\end{itemize}