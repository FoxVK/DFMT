\section{Tuner}
\label{sec:tuner}
Jak jsem zmínil v kapitole \ref{subsec:konstrukce}
\begin{itemize}
\item nepoužití přerušení a důsledky - nutnost polingu
\item nějak vkusně s ohledem na předchozí kapitolu vyzdvnhnout použítí \iic - jendou větou na začátek
\end{itemize}


\subsection{Zvukové rozhraní \iis}

\begin{table}[ht!]
\begin{center}
\begin{tabular}{|l|c|l|l|}
\hline 
MCU & Směr & Tuner & Význam \\ 
\hline 
SDI & $\Rightarrow$ & DOUT & Datový signál \\ 
\hline 
SCK & $\Rightarrow$ & DCLK & Hodiny \\ 
\hline
SS & $\Leftarrow$ & DFS & Signál určení kanálu \\ 
\hline 
\end{tabular}
\end{center}
\caption{Popis \iis signálů.}
\label{tab:iis_signals} 
\end{table}


Rozhraní \iis je v případě jednosměrného přenosu v podstatě rozhraní SPI doplněné o další datový signál. Význam signálů včetně odpovídajících názvů pinů na MCU a tuneru shrnuje tabulka \ref{tab:iis_signals}. V případě \iis se nejedná o sběrnici. Vždy komunikují právě dvě zařízení kde jedno je master a druhé slave. Master vždy vysílá datový signál a signál přepínání kanálů. Datový signál vysílá samozřejmě zřízení, které je zdrojem zvuku. Je možný i obousměrný přenos, potom je nutné použití dvou datových linek.

Tuner umí pracovat pouze v režimu slave, takže hodinový signál a signál přepínání kanálů je generován z MCU. Vzorkování zvuku v tuneru je řízeno jeho vlastním interním oscilátorem. Díky tomuto pravděpodobně občas dojde k zahození některého vzorku pokud MCU čte pomaleji než tuner provádí vzorkování, a nebo dojde k přečtení náhodných dat v opačném případě. Nicméně v praxi jsem s tímto nezaznamenal žádný problém.  

\InsertFigure{figures/i2s_format.png}{\textwidth}{Časový diagram \iis přenosu. (Převzato z \cite{iis})}{fig:iis-diagram}

Vyjma standardního formátu dat popsaného v \iis specifikace \cite{iis} vzniklo ještě několik dalších formátů. Použil jsem standardní formát, tak jak je zobrazen na obrázku \ref{fig:iis-diagram}.

TODO:
\begin{itemize}
\item Umravnit pozicování obrázku
\item Problém synchronizace hodin - presunout do USB jako samostatnou podkapitolu a nebo spáchat samostatnou kapitolu přeposílání zvuku
\end{itemize}


\subsection{Ovládací rozhraní \iic}
Ovláddání tuneru se provádí přes rozhraní \iic. Toto rozhraní je součásti drtivé většiny mikrokontrolérů, není ho třeba blíže popisovat. Tuner se do režimu \iic přepne připojením pinů GPO1 na 1 GPO2 na 0.

\begin{table}[ht!]
\begin{center}
\begin{tabular}{|c|c|c|}
\hline 
SEN & Adresa čtení & Adresa zápisu  \\ 
\hline 
0 & 0x23 & 0x22 \\ 
\hline 
1 & 0xC7 & 0xC6 \\ 
\hline 
\end{tabular}
\end{center}
\caption{\iic adresy.}
\label{tab:iic_addresses} 
\end{table}

Základní informací nezbytnou pro komunikace s tunerem je \iic adresa. Tuner umožňuje změnu adresy pomocí změny úrovně na pinu SEN. Adresy jsou v tabulce \ref{tab:iic_addresses}.

Tuner se ovládá zapisováním příkazů a případným čtením odpovědí. Dále je zde dvojice příkazů pro čtení a zápis proměnných. (properties) tuneru. Popis všech příkazů a proměnných je v programovací příručce k tuneru \cite{tuner-programing}.


TODO:
\begin{itemize}
\item minmálně obecný formát příkazů
\item čtení a zápis se nikdy neprolíná
\item popsat že se ovládá pomocí příkazů a properties
\item a přepsat sem použité příkazy ?
\item možná přidat kapitolku se spuštěním tuneru - možná jenom flowchart a možná ho hodit do přílohy - nechme to na konec kdyby nevycházel rozsah tak se popíše textově
\end{itemize}


\subsubsection{Čtení RDS z tuneru}
TODO:
\begin{itemize}
\item vyčtení z tuneru
\item dekódování základních informací- třeba radio text
\end{itemize}