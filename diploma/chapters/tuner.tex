\section{Tuner}
\label{sec:tuner}
Jak jsem zmínil v kapitole \ref{subsec:konstrukce} k přijmu rozhlasového vysílání je použit integrovaný obvod SI4735-D60. Zvuk je do MCU přenášen přes rozhraní \iis, samotné ovládání tuneru je realizováno přes sběrnici \iic.

\subsection{Zvukové rozhraní \iis}

\begin{table}[ht!]
\begin{center}
\begin{tabular}{|l|c|l|l|}
\hline 
MCU & Směr & Tuner & Význam \\ 
\hline 
SDI & $\Rightarrow$ & DOUT & Datový signál \\ 
\hline 
SCK & $\Rightarrow$ & DCLK & Hodiny \\ 
\hline
SS & $\Leftarrow$ & DFS & Signál určení kanálu \\ 
\hline 
\end{tabular}
\end{center}
\caption{Popis \iis signálů.}
\label{tab:iis_signals} 
\end{table}


Rozhraní \iis je v případě jednosměrného přenosu v podstatě rozhraní SPI doplněné o další datový signál. Význam signálů včetně odpovídajících názvů pinů na MCU a tuneru shrnuje tabulka \ref{tab:iis_signals}. V případě \iis se nejedná o sběrnici. Vždy komunikují právě dvě zařízení kde jedno je master a druhé slave. Master vždy vysílá datový signál a signál přepínání kanálů. Datový signál vysílá samozřejmě zřízení, které je zdrojem zvuku. Je možný i obousměrný přenos, potom je~nutné použití dvou datových linek.

Tuner umí pracovat pouze v režimu slave, takže hodinový signál a signál přepínání kanálů je~generován z MCU. Vzorkování zvuku v tuneru je řízeno jeho vlastním interním oscilátorem. Díky tomuto pravděpodobně občas dojde k zahození některého vzorku pokud MCU čte pomaleji než tuner provádí vzorkování, a nebo dojde k přečtení náhodných dat v opačném případě. Nicméně v praxi jsem s tímto nezaznamenal žádný problém.  

\InsertFigure{figures/i2s_format.png}{120mm}{Časový diagram \iis přenosu. (Převzato z \cite{iis})}{fig:iis-diagram}

Vyjma standardního formátu dat popsaného v \iis specifikace \cite{iis} vzniklo ještě několik dalších formátů. Použil jsem standardní formát, tak jak je zobrazen na obrázku \ref{fig:iis-diagram}.


\subsection{Ovládací rozhraní \iic}
Ovládání tuneru se provádí přes rozhraní \iic. Toto rozhraní je součástí drtivé většiny mikrokontrolérů, není ho třeba blíže popisovat. Tuner se do režimu \iic přepne připojením pinů GPO1 na napájecí napětí a GPO2 k zemi.

\begin{table}[ht!]
\begin{center}
\begin{tabular}{|c|c|c|}
\hline 
SEN & Adresa čtení & Adresa zápisu  \\ 
\hline 
0 & 0x23 & 0x22 \\ 
\hline 
1 & 0xC7 & 0xC6 \\ 
\hline 
\end{tabular}
\end{center}
\caption{\iic adresy.}
\label{tab:iic_addresses} 
\end{table}

Základní informací nezbytnou pro komunikace s tunerem je \iic adresa. Tuner umožňuje změnu adresy pomocí změny úrovně na pinu SEN. Adresy jsou v tabulce \ref{tab:iic_addresses}.

\subsection{Ovládání tuneru}
\label{subsec:tuner-control}
Tuner se ovládá zapisováním příkazů a případným čtením odpovědí. Dále je zde dvojice příkazů pro čtení a zápis proměnných. (properties) tuneru. Kompletní popis všech příkazů a proměnných je v programovací příručce k tuneru \cite{tuner-programing}.

\begin{table}[ht!]
\begin{center}
\begin{tabular}{|l|c|l|l|}
\hline 
Název pole & Délka &  \\ 
\hline
CMD & 1 B & Příkaz.\\
\hline
ARG1 - ARG7 & 0 - 7 B & Případný argument příkazu.\\
\hline
\end{tabular} 
\end{center}
\caption{Formát příkazu pro tuner.}
\label{tab:tuner-cmd} 
\end{table}

\begin{table}[ht!]
\begin{center}
\begin{tabular}{|l|c|l|l|}
\hline 
Název pole & Délka &  \\ 
\hline
STATUS & 1 B & Status tuneru.\\
\hline
RESP1 - RESP15 & 0 - 15 B & Případná další data odpovědi.\\
\hline
\end{tabular} 
\end{center}
\caption{Formát odpovědi tuneru.}
\label{tab:tuner-rpl} 
\end{table}

Všechny příkazy mají jednotný formát vyobrazený v~tabulce \ref{tab:tuner-cmd}. Každý příkaz je jednou transakcí zápisu na \iic sběrnici. Případnou odpověď na příkaz je možné získat následnou transakcí čtení z \iic sběrnice. Formát odpovědi je v tabulce \ref{tab:tuner-rpl}. Jak je patrné odpověď na každý příkaz obsahuje vždy minimálně jeden byte se stavem zařízení (STATUS). Význam jednotlivých bitů v~tomto byte je následující:

\begin{itemize}
\item Bit 7 \textbf{CTS} - Pokud je v 1 je možné do tuneru odeslat další příkaz.
\item Bit 6 \textbf{ERR} - Pokud je v 1 signalizuje chybu provádění příkazu.
\item Bity 5 - 3 \textbf{Rezervováno} - Hodnoty těchto bitů nejsou specifikovány.
\item Bit 2 \textbf{RDSINT} - Pokud je v 1 signalizuje vznik přerušení od přijmu RDS.
\item Bit 1 \textbf{ASQINT} - Pokud je v 1 signalizuje dokončení měření parametrů přijmu.
\item Bit 0 \textbf{STCINT} - Pokud je v 1 signalizuje dokončení ladění nebo vyhledávání stanice.
\end{itemize}  

Bity CTS a ERR se aktualizují vždy. Pro čtení nejnižších třech bitů je potřeba aktualizovat jejich hodnoty příkazem GET\_INT\_STATUS. Tento příkaz má číslo 0x14 a nemá žádné argumenty. Tuner umožňuje nakonfigurovat signalizaci změny některého z těchto bitů změnou úrovně na výstupu GPO2/INT. Této možnosti jsem se ale rozhodl nevyužít.

Jak jsem zmínil výše, kromě příkazů se parametry tuneru nastavují a nebo čtou pomocí proměnných. jedná se vždy o 16-ti bitové hodnoty, které se identifikují taktéž 16-ti bitovým číslem. 

Zápis hodnoty se provádí příkazem SET\_PROPERTY (0x12), za kterým následuje jeden nulový byte, za ním horní a poté dolní byte čísla proměnné. Dále se vyšle horní byte a poté dolní byte vlastní hodnoty. 

Čtení se provádí obdobně, příkazem GET\_PROPERTY (0x14) za kterým se vyšle nulový byte následovaný horním a dolním byte čísla zapisované proměnné. Hodnota se získá následným čtením z \iic sběrnice. Horní byte hodnoty je až ve druhém bytu argumentu ARG-2 (viz. tab.~\ref{tab:tuner-rpl}), dolní byte v je v ARG-3.

%TODO přidat tabulky pro property ?

\subsubsection{Inicializace tuneru}
\label{subsubsec:tun-init}
\InsertFigure{figures/tuner-init.eps}{111mm}{Diagram inicializace tuneru.}{fig:tuner-init}

Po spuštění MCU vždy provádí inicializaci obou tunerů. Průběh inicializace shrnuje diagram~\ref{fig:tuner-init}. Z důvodu zkrácení doby spuštění modulu probíhá inicializace formou jednoduchého "multitaskingu". Pokud je jeden tuner zaneprázdněn testuje se stav druhého tuneru a v případě, že~není zaneprázdněn dojde k vyslání příkazu, pokud je zaneprázdněn testuje se znova první  tuner a~tak~dále. Ze stejného důvodu je v příkazu FM\_TUNE\_STATUS nastaven příznak zrychleného nepřesného ladění.


\begin{table}[ht!]
\begin{center}
\begin{tabular}{|c|c|c|c|c|c|c|c|c|}
\hline 
Bit & 7. & 6. & 5. & 4. & 3. & 2. & 1. & 0. \\ 
\hline 
CMD & \multicolumn{8}{c|}{0x01} \\ 
\hline 
ARG1 & CTSIEN = 0 & GPO2OEN = 0 & PATCH = 0 & XOSCEN = 0 & \multicolumn{4}{c|}{FUNC = 0} \\ 
\hline 
ARG2 & \multicolumn{8}{c|}{OPMODE = 0b10110000} \\ 
\hline 
\end{tabular} 
\end{center}
\caption{Příkaz spuštění tuneru POWER\_UP.}
\label{tab:tuner-power-up} 
\end{table}

Inicializace začíná spuštěním tuneru příkazem POWER\_UP. Formát příkazu je v tabulce \ref{tab:tuner-power-up}. Součástí příkazu jsou tyto parametry:

\begin{itemize}
\item \textbf{CTSIEN} - Možnost hardwarové signalizace přerušení v okamžiku kdy je možné odeslat další příkaz. Hardwarové přerušení nepoužívám.
\item \textbf{GPO2OEN} - Nastavení pinu GPO2 jako výstupu.
\item \textbf{PATCH} - Umožňuje načtení novějšího firmware do integrovaného obvodu tuneru.
\item \textbf{XOSCEN} - Použití krystalu jako zdroje hodinové frekvence. Místo krystalu používám přímo hodinový kmitočet rozhraní \iis.
\item \textbf{FUNC} - Použitá hodnota nastaví režim přijmu FM rádia.
\item \textbf{OPMODE} - Nastavení výstupu zvuku. Význam hodnoty v tabulce určuje, že bude použit pouze digitální výstup zvuku \iis.
\end{itemize}

Odpovědí na tento příkaz je pouze jeden byte se statusem tuneru.

Dále je potřeba nastavit dvě proměnné. REFCLK\_PRESCALE hodnotu děličky hodinového kmitočtu a do REFCLK\_FREQ výslednou frekvenci, která se musí pohybovat v rozsahu $ 31,130 - 34,406 Khz $. Jak už jsem zmínil jako zdroj hodinové frekvence je použita hodinová frekvence \iis rozhraní. Tato frekvence vyplývá ze součinu počtu kanálů, jejich rozlišení a vzorkovací frekvence. $ 2 \cdot 16 \cdot 48 = 1536 kHz $. Dělící poměr tedy vychází $ 1536 / 32 = 48 $ při frekvenci $ 32000 kHz $.

Zatímco REFCLK\_FREQ obsahuje pouze 16-ti bitovou hodnotu frekvence v Hz v REFCLK\_PRESCALE jsou pro hodnotu dělícího poměru vyhrazeny bity 0-11. Bity 13-15 musí být nulové, bit 12 je nastaven na hodnotu 1, což znamená, že se jako zdroj hodinové frekvence použije \iis.

Poslední proměnná, kterou je potřeba nastavit, je DIGITAL\_OUTPUT\_SAMPLERATE. Obsahuje hodnotu vzorkovací frekvence. V tomto případě je to hodnota 48000.

\begin{table}[ht!]
\begin{center}
\begin{tabular}{|l|c|c|}
\hline 
Název & Číslo & Hodnota \\ 
\hline 
REFCLK\_FREQ & 0x0201 & 0x7D00 \\ 
\hline 
REFCLK\_PRESCALE & 0x0202 & 0x102A \\ 
\hline 
DIGITAL\_OUTPUT\_SAMPLE\_RATE & 0x0104 & 0xBB80 \\ 
\hline 
\end{tabular} 
\end{center}
\caption{Proměnné použité pro inicializaci tuneru.}
\label{tab:tuner-init-prop} 
\end{table}

Proměnné se nastavují pomocí příkazu SET\_PROPERTY popsaného v kapitole \ref{subsec:tuner-control}. Výsledné hodnoty a čísla proměnné jsou shrnuty v tabulce \ref{tab:tuner-init-prop}.

\begin{table}[ht!]
\begin{center}
\begin{tabular}{|c|c|c|c|c|c|c|c|c|}
\hline 
Bit & 7. & 6. & 5. & 4. & 3. & 2. & 1. & 0. \\ 
\hline 
CMD & \multicolumn{8}{c|}{0x20} \\ 
\hline 
ARG1 & \multicolumn{6}{c|}{0x00} & FREEZE = 0 & FAST = 1 \\ 
\hline 
ARG2 & \multicolumn{8}{c|}{FREQ$_{{H}}$ = 0x24} \\ 
\hline 
ARG3 & \multicolumn{8}{c|}{FREQ$_{{L}}$ = 0x94} \\ 
\hline 
ARG4 & \multicolumn{8}{c|}{ANTCAP = 0x00} \\ 
\hline 
\end{tabular} 
\end{center}
\caption{Příkaz spuštění tuneru FM\_TUNE\_FREQ.}
\label{tab:tuner-tune-freq} 
\end{table}

Poslední částí inicializace tuneru je naladění frekvence 93,7 MHz příkazem FM\_TUNE\_FREQ. Formát příkazu včetně použitých hodnot je v tabulce \ref{tab:tuner-tune-freq}. Význam parametrů je následující:

\begin{itemize}
\item \textbf{FREEZE} - Nastavení způsobí pozvolný přechod zvuku po přeladění.
\item \textbf{FAST} - Nastavení způsobí rychlé ale nepřesné přeladění.
\item \textbf{FREQ$_{{H}}$} - Horní byte frekvence v desetinách MHz.
\item \textbf{FREQ$_{{L}}$} - Dolní byte frekvence v desetinách MHz.
\item \textbf{ANTCAP} - Nastavení kapacity vstupního kondenzátoru antény. Hodnota 1-191 pF. Hodnota 0 znamená automatické nastavení.
\end{itemize}

Odpovědí na tento příkaz je pouze jeden byte obsahující status tuneru. Dokončení ladění je signalizováno nastavením bitu STCINT ve statusu. K aktualizaci hodnoty tohoto bytu je nutné vždy vyslat příkaz GET\_INT\_STATUS. Tento příkaz se skládá z jediného byte s číslem příkazu 0x14. Odpovědí na tento příkaz je rovněž jediný byte se statusem tuneru kde jsou aktualizovány hodnoty bitů signalizujících přerušení RDSINT, ASQINT a STCINT.


\begin{table}[ht!]
\begin{center}
\begin{tabular}{|c|c|c|c|c|c|c|c|c|}
\hline 
Bit & 7. & 6. & 5. & 4. & 3. & 2. & 1. & 0. \\ 
\hline 
CMD & \multicolumn{8}{c|}{0x22} \\ 
\hline 
ARG1 & \multicolumn{6}{c|}{0x00} & CANCEL = 0 & INTACK = 1 \\ 
\hline 
\end{tabular} 
\end{center}
\caption{Příkaz zjištění stavu ladění FM\_TUNE\_STATUS.}
\label{tab:tuner-tune-status} 
\end{table}

\begin{table}[ht!]
\begin{center}
\begin{tabular}{|c|c|c|c|c|c|c|c|c|}
\hline 
Bit & 7. & 6. & 5. & 4. & 3. & 2. & 1. & 0. \\ 
\hline 
SATUS & CTS & ERR & X & X & RSQINT & RDSINT & X & STCINT \\ 
\hline 
RESP1 & BLTF & X & X & X & X & X & AFCRL & VALID \\ 
\hline 
RESP2 & \multicolumn{8}{c|}{READFREQ$_{{H}}$} \\ 
\hline 
RESP3 & \multicolumn{8}{c|}{READFREQ$_{{L}}$} \\ 
\hline 
RESP4 & \multicolumn{8}{c|}{RSSI} \\ 
\hline 
RESP5 & \multicolumn{8}{c|}{SNR} \\ 
\hline 
RESP6 & \multicolumn{8}{c|}{MULT} \\ 
\hline 
\end{tabular} 
\end{center}
\caption{Odpověď na příkaz zjištění stavu ladění FM\_TUNE\_STATUS.}
\label{tab:tuner-tune-status-resp} 
\end{table}

Ke smazání bitu STCINT je použit příkaz FM\_TUNE\_STATUS viz. tabulka \ref{tab:tuner-tune-status}. Kromě smazání tohoto bitu nastavením parametru INTACK je možné nastavením tohoto parametru zrušit probíhající ladění nebo vyhledávání stanice. Jak je vidět z tabulky \ref{tab:tuner-tune-status-resp}, odpověď na příkaz obsahuje kromě statusu tuneru následující informace:

\begin{itemize}
\item \textbf{BLTF} - Je nastaven pokud vyhledávání stanice přeteklo přes hodnotu maximální nebo podteklo hodnotu minimální laditelné frekvence.
\item \textbf{AFCRL} - Je nastaven pokud je automatické dolaďování aktivní.
\item \textbf{VALID} - Naladěná frekvence byla vyhodnocena jako validní. 
\item \textbf{READFREQ$_{{H}}$} - Horní byte naladěné frekvence v desetinách MHz.
\item \textbf{READFREQ$_{{L}}$} - Dolní byte naladěné frekvence v desetinách MHz.
\item \textbf{RSSI} - Indikátor síly přijímaného signálu v dB$\mu$V.
\item \textbf{SNR} - Odstup signálu od šumu v dB.
\item \textbf{MULT} - Indikátor míry odrazů v signálu.
\end{itemize}

\subsubsection{Vyhledávání stanic}
\begin{table}[ht!]
\begin{center}
\begin{tabular}{|c|c|c|c|c|c|c|c|c|}
\hline 
Bit & 7. & 6. & 5. & 4. & 3. & 2. & 1. & 0. \\ 
\hline 
CMD & \multicolumn{8}{c|}{0x22} \\ 
\hline 
ARG1 & \multicolumn{4}{c|}{0x00} & SEEKUP & WRAP & \multicolumn{2}{c|}{0x00} \\ 
\hline 
\end{tabular} 
\end{center}
\caption{Příkaz zjištění stavu ladění FM\_TUNE\_STATUS.}
\label{tab:tuner-seek} 
\end{table}

Vyhledávání stanic se provádí pomocí příkazu FM\_SEEK\_START popsaném v tabulce \ref{tab:tuner-seek}. Příkaz lze parametrizovat pouze pomocí dvou bitů. Nastavením bitu SEEKUP se bude vyhledávat směrem k vyšším frekvencím. Nastavením bitu WRAP bude při dosažení nejvyšší frekvence vyhledávání pokračovat znova od nejnižší frekvence, v případě sestupného vyhledávání se bude po dosažení nejnižší frekvence pokračovat od horní frekvence. Pokud bit WRAP není nastaven, při dosažení nejnižší nebo nejvyšší frekvence dojde k zastavení vyhledávání.

Ukončení vyhledávání je signalizováno nastavením bitu STCINT ve statusu odpovědi tuneru. Detekce ukončení vyhledávání se provádí shodně jako detekce ukončení ladění popsaná v kap. \ref{subsubsec:tun-init}. Naladěnou frekvenci a informace o kvalitě signálu je možné přečíst příkazem FM\_TUNE\_STATUS, který byl taktéž popsán v kapitole \ref{subsubsec:tun-init}. Tyto informace se vždy aktualizují po naladění, opětovné volání příkazu FM\_TUNE\_STATUS vrací vždy stejné hodnoty. 


\subsubsection{Čtení RDS z tuneru}
%TODO možná obrázek
RDS je rozšíření pozemního rozhlasového vysílání o souběžné vysílání různých digitálních informací. Z nejznámějších informací je to RT (Radio Text) - přenos až 64 znaků textu, typicky s názvem právě přehrávané skladby a nebo názvem rádiové stanice. Potom AF  (Alternative frequency) seznam dalších frekvencí, na kterých je možné stanici naladit, což lze s využitím informace PI (Programme Identification), který stanici jednoznačně identifikuje k automatickému přelaďování na frekvenci s lepším příjmem. Kompletní popis je v \cite{rds}.

Proud přenášených dat se dělí do třiceti skupin označených 1A, 1B, 2A ... 15A, 15B. Každá skupina má pevně danou velikost 104 bitů a skládá se ze čtyř stejně dlouhých bloků. Chtěl bych upozornit, že v normě \cite{rds} jsou tyto bloky označeny čísly 1 až 4, kdežto v programovací příručce tuneru \cite{tuner-programing} jsou označeny písmeny A až D. Každý blok nese 16~bitů užitečných dat a~10-ti~bitové kontrolní slovo. Vyhodnocování chyb v přijatých datech a jejich případnou korekci zajišťuje tuner. Pro příjem je v tuneru k dispozici fronta na 14 skupin.

\begin{table}[ht!]
\begin{center}
\begin{tabular}{|c|c|c|c|c|c|c|c|c|c|}
\hline 
Bit & 15. & 14. & 13. & 12. & 11. & 10. & 9. & 8. \\  
\hline 
Horní byte & \multicolumn{2}{c|}{BLETHA = 2} & \multicolumn{2}{c|}{BLETHB = 2} & \multicolumn{2}{c|}{BLETHC = 2} & \multicolumn{2}{c|}{BLETHD = 2} \\ 
\hline 
Bit & 7. & 6. & 5. & 4. & 3. & 2. & 1. & 0. \\
\hline 
Dolní byte & \multicolumn{7}{c|}{0x00} & RDSEN = 1 \\ 
\hline 
\end{tabular} 
\end{center}
\caption{Proměnná konfigurace příjmu RDS FM\_RDS\_CONFIG.}
\label{tab:tuner-rds-config} 
\end{table}

Příjem RDS v tuneru je potřeba nejprve nakonfigurovat a povolit. Díky nepoužití hardwarového přerušení, je možné konfiguraci přerušení přeskočit. Konfigurace přijmu RDS se zjednoduší na nastavení jediné proměnné FM\_RDS\_CONFIG, jejíž formát se nachází v tabulce \ref{tab:tuner-rds-config}. Nastavením bitu RDSEN této proměnné dojde k povolení přijmu RDS. Pole BLETHA až BLETHD specifikují dovolenou míru chybovosti bloků A až D přijímaných skupin a to takto:

\begin{itemize}
\item 0 - Pokud je v bloku jakákoliv chyba celá skupiny bude zahozena.
\item 1 - Skupina bude přijata, pokud v bloku došlo k opravě nejvýše dvou bitů. 
\item 2 - Skupina bude přijata, pokud v bloku došlo k opravě nejvýše pěti bitů.
\item 3 - Jakékoliv, i neopravitelné, množství chyb v bloku nezpůsobí zahození skupiny.
\end{itemize}

\begin{table}[ht!]
\begin{center}
\begin{tabular}{|c|c|c|c|c|c|c|c|c|}
\hline 
Bit & 7. & 6. & 5. & 4. & 3. & 2. & 1. & 0. \\ 
\hline 
CMD & \multicolumn{8}{c|}{0x24} \\ 
\hline 
ARG1 & \multicolumn{5}{c|}{0x00} & STATUSONLY = 0 & 0x00 & INTACK = 1 \\ 
\hline 
\end{tabular} 
\end{center}
\caption{Příkaz přečtení přijaté RDS skupiny FM\_RDS\_STATUS.}
\label{tab:tuner-rds-get} 
\end{table}

Vyčítání přijatých skupin se potom provádí příkazem FM\_RDS\_STATUS, uvedeným v tabulce \ref{tab:tuner-rds-get}. Nastavením bitu STATUSONLY v argumentu tohoto příkazu je možné pouze vyčíst informace o stavu příjmu RDS bez odebrání nejstarší přijaté skupiny z fronty. Nastavením druhého bitu INTACK dojde ke smazání příznaku přerušení RDSINT ve statusu tuneru. Vzhledem k nepoužití přerušení nastavení bitu INTACK nemá význam.

\begin{table}[ht!]
\footnotesize 
\begin{center}
\begin{tabular}{|c|c|c|c|c|c|c|c|c|}
\hline 
Bit & 7. & 6. & 5. & 4. & 3. & 2. & 1. & 0. \\ 
\hline 
SATUS & CTS & ERR & X & X & RSQINT & RDSINT & X & STCINT \\ 
\hline 
RESP1 & X & X & \makecell{RDSNEW\\BLOCKB} & \makecell{RDSSYNC\\BLOCKA} & X & \makecell{RDSSYNC\\FOUND} & \makecell{RDSSYNC\\LOST} & RDSRECV \\ 
\hline
RESP2 & X & X & X & X & X & GRPLOST & X & RDSSYNC \\
\hline 
RESP3 & \multicolumn{8}{c|}{RDSFIFOUSED} \\ 
\hline 
RESP4 & \multicolumn{8}{c|}{BLOCKA$_{{H}}$} \\ 
\hline 
RESP5 & \multicolumn{8}{c|}{BLOCKA$_{{L}}$} \\
\hline 
RESP6 & \multicolumn{8}{c|}{BLOCKB$_{{H}}$} \\ 
\hline 
RESP7 & \multicolumn{8}{c|}{BLOCKB$_{{L}}$} \\ 
\hline 
RESP8 & \multicolumn{8}{c|}{BLOCKC$_{{H}}$} \\ 
\hline 
RESP9 & \multicolumn{8}{c|}{BLOCKC$_{{L}}$} \\ 
\hline 
RESP10 & \multicolumn{8}{c|}{BLOCKD$_{{H}}$} \\ 
\hline 
RESP11 & \multicolumn{8}{c|}{BLOCKD$_{{L}}$} \\ 
\hline 
RESP12 & \multicolumn{2}{c|}{BLEA} & \multicolumn{2}{c|}{BLEB} & \multicolumn{2}{c|}{BLEC} & \multicolumn{2}{c|}{BLED} \\ 
\hline 
\end{tabular} 
\end{center}
\caption{Odpověď na příkaz přečtení přijaté RDS skupiny FM\_RDS\_STATUS.}
\label{tab:tuner-rds-get-resp} 
\end{table}

Odpověď na tento příkaz v tabulce \ref{tab:tuner-rds-get-resp} je podle očekávání poněkud obsáhlejší. První byte obstahuje jako vždy status tuneru. V následujících dvou bytech za statusem jsou bity jejihž LOG 1 má následující význam:

\begin{itemize}
\item \textbf{RDSNEWBLOCKB, RDSNEWBLOCKA} - Byl přijat validní blok B nebo A.
\item \textbf{RDSSYNCFOUND} - Příjem RDS je synchronizován.
\item \textbf{RDSSYNCLOST} - Ztracena synchronizace příjmu RDS, např. z důvodu špatného signálu.
\item \textbf{RDSRECV} - Nastaven pokud je ve frontě počet skupin větší nebo roven hodnotě proměnné FM\_RDS\_INT\_FIFO\_COUNT. Tato proměnná byla ponechána ve výchozím nastavení na hodnotě 0, takže tento bit bude nastaven vždy.
\item \textbf{GRPLOST} - Přetečení fronty přijatých skupin.
\item \textbf{RDSSYNC} - Příjem RDS je synchronizován. (Aktuální stav.)
\end{itemize}

V bytu RDSFIFOUSED je počet skupin ve frontě. Pokud je tato hodnota větší než nula, v~následujících osmi bytech se nachází data nejstarší přijaté skupiny. V posledním byte odpovědi jsou informace o chybovosti přijmu jednotlivých bloků skupiny BLEA až BLED. Význam hodnot chybovosti je tento:

\begin{itemize}
\item 0 - Blok přijat bez chyby.
\item 1 - V bloku došlo k opravě jednoho až dvou bitů. 
\item 2 - V bloku došlo k opravě tří až pěti bitů.
\item 3 - Blok nebylo možné opravit. Neobsahuje platná data.
\end{itemize} 

Aby nedoházelo k přetečení fronty přijatých skupin, je potřeba frontu vyčítat v dostatečně krátkých intervalech. Zde je možné využít z pevné délky skupiny 104 b a rychlosti přenosu 1187,5 kHz. Přenos skupiny tedy trvá 87,5 ms. K naplnění 14-ti prvkové fronty tedy dojde nejdříve za~1,2~s. Pokud bude interval vyčítání vždy kratší, k přetečení fronty nedojde.
