\section{Závěr}
\label{sec:Conclusion}

\InsertFigure{figures/breadborad-landscape.jpg}{123mm}{Modul v kontaktním poli.}{fig:breadborad}

Cílem této práce byl návrh modulu USB FM přijímače, který bude v operačním systému reprezentován jako zvuková karta. Přijímač má obsahovat dva tunery, kdy jeden bude používán pro~přehrávání a druhý pro vyhledávání dalších stanic. 

Prvními dvěma úkoly byl výběr vhodných součástek a ověření funkce na kontaktním poli. Toto je rozebráno v kapitole \ref{sec:Vyber}. Byly vybrány tunery ve formě integrovaných obvodů SI4735-D60, které budou na sběrnici USB připojeny prostřednictvím mikrokontroléru PIC32MX250F128B. Výslednou konstrukci nejlépe ilustruje obrázek \ref{fig:tuner-block} na straně \pageref{fig:tuner-block}. Do mikrokontroléru byl napsán firmware s vlastním zjednodušeným ovladačem USB. Nad tímto ovladačem běží USB - \iic tunel, pomocí kterého jsou ovládány tunery a rozhraní, které přeposílá digitální zvuk z tunerů do USB. Funkčnost vzniklého zařízení jsem úspěšně ověřil na kontaktním poli (viz. obrázek \ref{fig:breadborad}). Nad rámec zadání jsme navíc navrhl  desku plošných spojů na které jsem zařízení úspěšně oživil (fotografie v příloze~\ref{sec:ap-pcb}). Zařízení má v této podobně mnohem kvalitnější příjem. Schéma zapojení je v příloze \ref{sec:ap-schema}. Zároveň je v elektronické verzi spolu s podklady pro výrobu DPS umístěn na~přiloženém CD.

Třetím úkolem bylo napsat knihovnu pro práci s tunerem pod OS Linux a Windows. Knihovna byla napsána v jazyce C s využitím multiplatformní knihovny libusb. Knihovna je rozdělena do třech vrstev a umožňuje tak snadné rozšíření funkcionality. Knihovna v současné podobně umožňuje přeladění vybraného tuneru na danou frekvenci, vyhledávání stanic a vyčtení parametrů naladěné stanice. Knihovna také byla doplněna o možnost příjmu RDS dat. Způsob jakým knihovna ovládání tunerů knihovnou je popsáno v kapitole \ref{sec:tuner}. Samotná knihovna je popsána v kapitole \ref{sec:knihovna}, samotné rozhraní knihovny je podrobně popsáno v dokumentaci vygenerované nástrojem Doxygen, která je umístěna na přiloženém CD.

Čtvrtým a poslední úkolem bylo napsat uživatelský program, který bude demonstrovat využití knihovny. Napsal jsem aplikaci s grafickým uživatelským rozhraním s využitím multiplatformního frameworku QT5. Tato aplikace umožňuje ladění frekvencí, vyhledávání stanic, zobrazuje parametry signálu naladěné frekvence a zobrazuje její RDS radio text. Zároveň jak~je~požadováno v úvodu zadání průběžně druhým tunerem vyhledává dostupně stanice a zobrazuje je~v~tabulce. Rozhraní nejlépe ilustruje obrázek \ref{fig:gui} na straně \pageref{fig:gui}.

%\subsection{Omezení}

%TODO:
%\begin{itemize}

%\item Nemožnost dostat se v suspendu do 0,5 mA
%\item Test mód neimplementován. Nutný pouze pro získání USB loga.

%suspend a test mod
%\end{itemize}
\FloatBarrier
\subsection{Možnosti dalšího rozšíření}
Hardware a částečně i firmware MCU je připraven pro přenos zvuku i z druhého tuneru. Vhodným doplnění deskriptoru USB konfigurace by bylo možno zařízení přepínat do režimu dvou zvukových karet. Poté by bylo možné nahrávat jinou stanici při současném přehrávání jiné a~nebo využít zařízení pro~streaming několika rozhlasových stanic.  

Na desce plošných spojů je připraveny pozice pro dvě tlačítka, která zatím nejsou osazena. Drobnou úpravou firmware by bylo možné využít je k vyhledávání stanic bez potřeby jakéhokoliv dalšího software, popřípadě stisky tlačítek posílat přes USB do aplikace a tam je dále vyhodnocovat.

Tunery umožňují příjem rozhlasu i na středních a dlouhých vlnách. Toto by vyžadovalo jednak úpravu knihovny, která tunery přepne do režimu pro příjem těchto pásem a také bude třeba hardware doplnit o vhodnou anténu spolu s transformátorem viz. katalogový list \cite{tuner-datasheet}.

Uživatelská aplikaci by bylo vhodné doplnit minimálně o možnost ovládat přehrávání zvuku a popřípadě jeho záznam.





\bigskip
\begin{flushright}
Bc. Pavel Kovář
\end{flushright}