\section{Úvod}
\label{sec:Uvod}
%Tento text je ukázkou sazby diplomové práce v \LaTeX{}u pomocí třídy dokumentů \verb|diploma|.

Cílem této diplomové práce je navrhnout USB FM přijímač rádiového vysílání. Přijímač bude v systému reprezentován jako zvuková karta a bude obsahovat dva tunery. Jeden bude sloužit k samotnému příjmu vysílání a druhý k vyhledávání dalších stanic. Dále napsat knihovnu, umožňující ovládání tuneru pod Operačními systémy Linux a Windows. Text této práce je členěn do třech kapitol.

První kapitola se zabývá konstrukcí modulu. Je zde shrnuto a~zhodnoceno několik možností jak realizovat jednak samotný příjem rozhlasového vysílání a~také několik možností řešení napojení tunerů na USB sběrnici.

Druhá kapitola je věnována popisu vybraného tuneru. Je zde uveden zvolený způsob získání zvukových dat, filozofie ovládání tuneru a popis všech použitých příkazů pro tuneru.

Ve třetí kapitole je rozebráno napojení tunerů na USB sběrnici. Je zde pois USB popis porblémů s harmony , chyby v křemíku použitého MCU, porblémy nesynchronizovaných hodin udesílání zvuku. Popsáno vzniklé USB rozhraní (TODO)

Čtvrtá kapitola se věnuje vzniklé knihovně a demonstračnímu programu. V první části kapitoly je popis filozofie a rozvrstvení knihovny včetně popisu funkcí. Doplněno ukázkovým kódem práce. Závěru kapitoly je popsána demonstrační aplikace. (TODO)

%V závěru jsou shrnuty dosažené cíle, vyzdvihnut vlastní přínos