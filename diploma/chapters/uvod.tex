\section{Úvod}
\label{sec:Uvod}
%Tento text je ukázkou sazby diplomové práce v \LaTeX{}u pomocí třídy dokumentů \verb|diploma|.

Cílem této diplomové práce je navrhnout USB FM přijímač rádiového vysílání. Přijímač bude v systému reprezentován jako zvuková karta a bude obsahovat dva tunery. Jeden bude sloužit k~samotnému příjmu vysílání a druhý k vyhledávání dalších stanic. Dále napsat knihovnu, umožňující ovládání tuneru pod operačními systémy Linux a Windows. Text této práce je členěn do~čtyř kapitol.

První kapitola se zabývá konstrukcí modulu. Je zde shrnuto několik možností jak realizovat jednak samotný příjem rozhlasového vysílání a~také několik možností řešení napojení tunerů na~USB sběrnici. Tyto možnosti jsou zhodnoceny a zvolené řešení je zde popsáno.

Druhá kapitola je věnována popisu vybraného tuneru. Je zde uveden zvolený způsob získání zvukových dat a možnosti ovládání tuneru. Kapitola obsahuje také popis všech použitých příkazů pro ovládání tuneru.

Ve třetí kapitole je rozebráno napojení tunerů na USB sběrnici. Je zde obecný popis USB a~popis neúspěšného řešení pomocí Harmony frameworku od firmy Michrocip. Je zde také upozorněno na chyby v samotném mikrokontroléru, které se řešení týkají. Součástí kapitoly je stručný popis USB protokolu v množství nezbytném pro zajištění přenosu zvuku a ovládání tunerů. V závěru kapitoly je popis protokolu pro vlastní ovládání tunerů a algoritmus řešící problém výpadků zvukových paketů v důsledku nezávislého časování čtení zvuku a jeho odesílání. 

Čtvrtá kapitola se věnuje vzniklé knihovně a aplikaci demonstrující její funkčnost. V první části kapitoly je popsána filozofie a rozvrstvení knihovny včetně popisu funkcí doplněného o~ukázkový kód základní práce s moduly přijímačů. Na konci kapitoly je popis demonstrační aplikace a způsoby překladu jak aplikace tak knihovny na obou požadovaných platformách.

%V závěru jsou shrnuty dosažené cíle, vyzdvihnut vlastní přínos